\section{Независимость событий}

\begin{example}\label{lect04:ex1}
	Студент из $N$ билетов выучил $n$. Перед ним $k$ человек. Какова вероятность, что студент вытащил "хороший"\, билет (то есть тот, который выучил)?
	
	
	$A = \{$студент вытащил хороший билет$\}$
	
	Ответ: $P(A) = \frac{n}{N}$.
\end{example}

\begin{theorem}\label{lect04:th1} Формула Байеса
	
	
	$ \Omega = \bigsqcup \limits_{i} B_i$, $P(B_i) > 0$ $\forall i$ $\implies \forall k$ $P(B_k \vert A) = \frac{P(A \vert B_k) P(B_k)}{\sum \limits_{i}  P(A\vert B_i) P(B_i)}$
\end{theorem}
\begin{proof}
	следствие формулы полной вероятности: 
	
	$$P(B_k \vert A) = \frac{P(AB_k)}{P(A)} = \frac{P(A \vert B_k) P(B_k)}{\sum \limits_{i}  P(A\vert B_i) P(B_i)}$$
\end{proof}
\begin{example}\label{lect04:ex2}
	При рентгеновском обследовании вероятность обнаружить заболевание N у больного равна 0.95, вероятность принять здорового человека за больного равна 0.05. Доля больных по отношению ко всему населению равна 0.01. Найти вероятность того, что человек здоров, если он был признан больным при обследовании. 
	
	
	\textit{Решение.}  Предположим, что:\\
	$A = $\{человек здоров\}, $B = $\{человек болен\},\\
	$B_1 =$ \{обнаружили заболевание\}, $B_2 =$\{не обнаружили заболевание\}.\\
	Тогда из условия получаем: 
	$$P(B_1 \vert B)=0.95,\, P(B_1 \vert A)=0.05,\, P(B)=0.01,\, P(A)=0.99.$$
	Вычислим полную вероятность признания больным:
	$$P(B_1) = P(B_1 \vert A)P(A)+P(B_2 \vert B)P(B) =  0.99 \cdot 0.05+0.01\cdot0.95=0.059.$$
	Вероятность «здоров» при диагнозе «болен» можно посчитать, применив формулу Байеса:
	$$P(A \vert B_1) =\frac{P(A)P(B_1 \vert A)}{P(A)P(B_1 \vert A) + P(B)P(B_1 \vert B)} = \frac{0.99 \cdot 0.05}{(0.99 \cdot 0.05+0.01\cdot 0.95}=0.839.$$
	
	Таким образом, $83.9\%$ людей, у которых обследование показало результат «болен», на самом деле здоровые люди. Удивительный результат возникает по причине значительной разницы в долях больных и здоровых. Болезнь $N$ — редкое явление, поэтому и возникает такой парадокс Байеса. При возникновении такого результата лучше всего сделать повторное обследование.
\end{example}
\begin{definition}\label{lect04:def1}
	Событие $A$ \textit{не зависит} от $B$, если $P(A \vert B) = P(A)$ (предполагается, что $P(B) \neq 0$).
\end{definition}
\begin{definition}\label{lect04:def2}
	События $A$ и $B$ \textit{независимые} (или $A$ не зависит от $B$ и  $B$ не зависит от $A$), если $P(AB) = P(A)P(B)$.
\end{definition}

\begin{prop}\label{lect04:prop1}
	A не зависит от B тогда и только тогда когда A и B независимы или B невероятно.
	%%как тут сформилировать утверждение???
\end{prop}
\begin{proof}
	$$
	\begin{array}{rcl}
	P(A \vert B) & = &  P(A) \\
	P(A \vert B)  & = & \frac{P(AB)}{P(B)} 
	\end{array} \iff P(A) = \frac{P(AB) }{P(B)} \iff P(AB) = P(A)P(B) $$
\end{proof}


\begin{example}\label{lect04:ex3}
	Из колоды, содержащей 36 карт, наудачу вынимается одна карта. 
	Пусть $A =$ \{достали туз\},  $B =$ \{достали карту трефовой масти $\clubsuit$\}.  Можно ли утверждать, что данные события независимы?
	
	
	\textit{Решение.}
	Из условия имеем: $|\Omega| = 36, \, |A| = 4, \, |B| = 9, \, |AB| = 1$.
	
	$$P(A) = \frac{4}{36} = \frac{1}{9}, \, P(B) = \frac{9}{36} = \frac{1}{4}, \, P(AB) = \frac{1}{36}$$
	Получаем, что $P(AB) = P(A)P(B) \implies$ по определению события независимы.
\end{example}
\begin{definition}\label{lect04:def3}
	События $A_z, \, z \in T$ называются \textit{попарно независимыми}, если $\forall s \neq t \in T$ $P(A_sA_t) = P(A_s)P(A_t).$
\end{definition}
\begin{definition}\label{lect04:def4}
	События $A_z, \, z \in T$ называются \textit{независимыми в совокупности}, если $\forall U \subset T$ $(|U| < |\infty|):$ $P(\bigcap \limits_{t\in U} A_t) = \prod \limits_{t\in U} P(A_t).$     
	
\end{definition}

\begin{example}\label{lect04:ex4}События попарно независимы, но зависимы с совокупности:
	Пусть $\Omega = \{1, 2, 3, 4\}, \, A_1 = \{1, 2\}, \, A_2 = \{1, 3\}, \, A_3 = \{1, 4\}.$\\
	Тогда $P(A_i) = \frac{2}{4} = \frac{1}{2}\,$ $\forall i = 1, \dots , 3$ и $P(A_iA_j) = \frac{1}{4}, \, i \neq j$.
	
	Получаем, что $P(A_iA_j) = \frac{1}{4} = \frac{1}{2} \cdot \frac{1}{2} = P(A_i)P(A_j) \implies$ события попарно независимы.\\
	$P(A_1A_2A_3) = \frac{1}{4} \neq \frac{1}{8} = P(A_1)P(A_2)P(A_3) \implies$ события зависимы в совокупности.
\end{example}

\begin{example}\label{lect04:ex5}  Пример Бернштейна 
	
	
	Рассмотрим правильный тетраэдр, три грани которого окрашены соответственно в красный, синий, зелёный цвета, а четвёртая грань содержит все три цвета. \\
	$A_1 = $  \{выпала грань, содержащая красный цвет\},\\
	$A_2 = $ \{выпала грань, содержащая синий цвет\},\\
	$A_3 = $ \{выпала грань, содержащая зеленый цвет\}.
	
\end{example}

\begin{lemma}\label{lect04:lemma1}
	Пусть $A_1, \dots, A_n$ $(n \geq 2)$ независимы в совокупности и $B_k = A_k$ или $A_k^C$ при $k = 1, \dots, n$. Тогда $B_1, \dots, B_n$ - независимые события.
\end{lemma}     
\begin{proof}
	Достаточно рассмотреть случай, когда $B_i = A_i^C$ и $B_k = A_k \, \forall k\neq i$. Проверим определение:  \\
	$U \subseteq \{1, \dots n \}$\\
	$S := \{1, \dots n\} \setminus U$\\
	$i \notin U \implies P(\bigcap \limits_{j \in U} B_j) = \prod \limits_{j \in U} P(B_j)$, потому что каждое $B_i = A_i$, а  $A_i$ - независимы в совокупности.\\
	$i \in U \implies P(\bigcap \limits_{j \in U} B_j) = P(\bigcap \limits_{j \in S} A_j) - P(A_i \cap (\bigcap \limits_{j \in S} A_j)) = P(\bigcap \limits_{j \in S} A_j) - P(A_i)P(\bigcap \limits_{j \in S} A_j)) = \prod \limits_{j \in S}P(A_j) - P(A_i) \prod \limits_{j \in S} P(A_j) = (1- P(A_i))\prod \limits_{j \in S} P(A_j) = P(A_i^C)\prod \limits_{j \in S} P(A_j) = \prod \limits_{j \in U} P(B_j)$
\end{proof}

\begin{theorem}\label{lect04:th2} Формула Эйлера из теории чисел
	
	Пусть $\phi : \N \to \N, \, n \mapsto \phi(n)$ - количество чисел $k \in \{1, \dots, n\}$, которые  взаимно простых с $n$. Тогда $\phi(n) = n \prod \limits_{p \in J_n} (1 - \frac{1}{p})$, где $J_n$ - множество всех простых делителей n (другими словами, если $n = p_1^{k_1} \dots p_m^{k_m}$, где простые числа $p_1 < \dots < p_m$ и $k_i \geq 1$ $i = 1, \dots, m$, $J_n = \{p_1, \dots, p_m\}$.
\end{theorem}    
\begin{proof}
	Введём $\Omega = \{1, \dots, n\}$, $\mathcal{F} = 2^\Omega$, пусть выполнено классическое определение вероятности $P(\{i\}) = \frac{1}{n}$ при $i = 1, \dots, n$ (можно сказать, что рассматривается случайный эксперимент, в котором наудачу выбирается одно число из множества от 1 до $n$).\\
	Пусть событие $A = \{$выбранное число взаимно просто с $n$\}. Тогда $P(A) = \frac{\phi(n)}{n}$. \\
	Введём событие $A_i =$\{выбранное число делится на $p_i$\} $= \{p_i, 2p_i, \dots, \frac{n}{p_i}p_i\}$, $|A_i| = \frac{n}{p_i}$, поэтому $P(A_i) = \frac{\frac{n}{p_i}}{n} = \frac{1}{p_i}$ для $i = 1, \dots, m$. \\
	Тогда $A_{i_1}\dots A_{i_k} =\{p_{i_1}\dots p_{i_k}, 2p_{i_1}\dots p_{i_k}, \dots, \frac{p_{i_1}\dots p_{i_k}}{n}p_{i_1}\dots p_{i_k}\}$ $(1 \leq i_1 < \dots < i_k \leq m) \implies P(A_{i_1}\dots A_{i_k}) = \frac{1}{p_{i_1}\dots p_{i_k}} = P(A_{i_k})\dots P(A_{i_k})$, то есть события $A_{i_1}, \dots, A_{i_k}$ - независимы.
	
	
	
	
	$$P(A) = P((\bigcup \limits_{i=1}^m A_i)^C) = P(\bigcap \limits_{i=1}^m A_i^C) \overset{\makebox[25pt]{\mbox{\normalfont\tiny по лемме}}}{=} \prod \limits_{i=1}^m P(A_i^C) = \prod \limits_{i=1}^m  (1 - P(A_i)) = \prod \limits_{i=1}^m (1 -\frac{1}{p_i})$$
\end{proof}


\begin{definition}\label{lect04:def5}
	Пусть даны системы событий $\mathcal{A}_t \subset \mathcal{F},\, t \in T$, где $T$ некоторое множество $|T| \geq 2$. Эти \textit{системы независимы в совокупности}, если для любого конечного $U \subset T$ независимы события $A_t \in \mathcal{A}_t, \, t \in U $.
\end{definition}

\begin{theorem}\label{lect04:th3}
	Пусть независимы $\pi-$системы $\mathcal{A}_t \subset \mathcal{F}, \, t \in T$, где $\mathcal{F}$ - это $\sigma-$алгебра. Тогда независимы  $\sigma \{\mathcal{A}_t\}$ $\sigma-$алгебры, порождённые этими системами.
\end{theorem}    
\begin{proof}
	Возьмём $\forall \{t_1, \dots, t_n \} \subset T$ и $A_{t_k} \in \mathcal{A}_{t_k}, \, k = 1, \dots, n$. Введём систему  $\mathcal{K}_{t_1}$, состоящую из событий $B$ таких, что $P(B A_{t_2}\dots A_{t_n}) = P(B)P(A_{t_2})\dots P(A_{t_m})$. 
	
	Легко увидеть, что $\mathcal{K}_{t_1}$ - это $\lambda$-система: 
	\begin{itemize}
		\item $\Omega \in \mathcal{K}_{t_1}$;
		\item $A, B \in \mathcal{K}_{t_1}, A\subseteq B \overset{?}{\implies} B \setminus A \in        \mathcal{K}_{t_1}$\\
		$P(B\dots A_{t_n}) - P(A\dots A_{t_m}) = (P(B) - P(A))P(A_{t_2})\dots P(A_{t_n}) = $\\
		$ = P(B\setminus A)P(A_{t_2})\dots P(A_{t_n}) = P((B\setminus A)A_{t_2} \dots A_{t_n})$;
		\item Следует из непрерывности меры.
	\end{itemize}
	
	По теореме о $\pi$-$\lambda$ - системах получаем, что $\sigma \{\mathcal{A}_{t_1} \} \subset \mathcal{K}_{t_1}$. Следовательно, соотношение $P(B A_{t_2}\dots A_{t_n}) = P(B)P(A_{t_2})\dots P(A_{t_m})$ верно для $\forall B \in \sigma\{\mathcal{A}_{t_1} \}$.
	
	
	Продолжая аналогичным образом, приходим к равенству $P(A_{t_1}\dots A_{t_n}) = P(A_{t_1})\dots P(A_{t_n})$ для $\forall A_{t_k} \in \sigma \{\mathcal{A}_{t_k} \}$.
\end{proof}


\begin{lemma}\label{lect04:lemma2} (о группировке)
	
	Пусть $\mathcal{A}_{t}, t \in T -$ семейство независимых $\sigma$-алгебр (то есть $\forall t \, \mathcal{A}_{t} \subset \mathcal{F}$). Возьмём $\Lambda \subset T$ такое, что $\Lambda = \bigsqcup \limits_{\alpha \in \Gamma} \Lambda_{\alpha}$. Определим $\mathcal{F}_{\alpha} = \sigma \{\mathcal{A}_t, \, t \in  \Lambda_{\alpha}\}$. Тогда $\mathcal{F}_{\alpha}, \, \alpha \in \Gamma$ - семейство независимых $\sigma$-алгебр.
\end{lemma}     
\begin{proof}
	$\forall n \in \N$ $\forall \alpha_1, \dots, \alpha_n \in \Gamma$\\
	$B_{\alpha_1} \in \mathcal{F}_{\alpha_1}, \dots, B_{\alpha_n} \in \mathcal{F}_{\alpha_n}$\\
	$P(B_{\alpha_1} \dots B_{\alpha_n}) \overset{?}{=} P(B_{\alpha_1})\dots P(B_{\alpha_n})$
	
	
	Пусть $\mathcal{F}_{\alpha_1} = \sigma \{\mathcal{A}_t, \, t \in  \Lambda_{\alpha_1}\}$, тогда $\mathcal{F}_{\alpha_1} = \sigma \{\overbrace{\mathcal{A}_{t_1} \cap \dots \cap \mathcal{A}_{t_m}}^{\makebox[32pt]{\mbox{\normalfont\tiny система из таких - $\pi$-система (обозн. $M_{\alpha_1}$)}}} | \forall t_1, \dots, t_m \in \Lambda_{\alpha_1} \}$ 
	
	
	
	$P(\overbrace{A_{t_1} \dots A_{t_n}}^{\in M_{\alpha_1}} \overbrace{B_{s_1} \dots B_{s_q}}^{\in M_{\alpha_2}}  \dots) = P(A_{t_1})\dots P(A_{t_m}) P(B_{s_1})\dots P(B_{s_q}) \dots =$\\
	$= P(A_{t_1}\dots A_{t_m}) P(B_{s_1} \dots B_{s_q}) \dots \implies \{M_{\alpha_i}\} - $ независимые $\pi$-системы $\implies$ незавимое семейство $\sigma$-алгебр, порожденных $\pi$-системами.  
	
\end{proof}


\begin{lemma}\label{lect04:lemma3} (Борель-Кантели)
	
	Пусть $A_n$ - последовательность событий
	\begin{enumerate} 
		\item Если $\sum P(A_k) < \infty$, то $P($произойдёт бесконечное число $A_k$-ых$) = 0$ 
		
		(или $P( \bigcap \limits_{n=1}^{\infty} \bigcup \limits_{k \geq n}^{\infty} A_k) = P(\lim \sup A_k)$).
		\item Если $A_1, A_2, \dots$ - независимы, то $P($произойдёт бесконечное число $A_k$-ых$) = 1$,
	\end{enumerate}
\end{lemma}        
\begin{proof}
	\begin{enumerate} 
		\item $A \subset \bigcup \limits_{k \geq n} A_k \, \forall n$, $A = \lim \sup A_k$\\
		$0 \leq P(A) \leq P(\bigcup \limits_{k \geq n} A_k) \leq \sum \limits_{k=n}^{\infty} P(A_k) \to 0$
		\item Проверим, что $P((\lim \sup A_k)^C) = 0$\\
		$A^C = (\bigcap \limits_{n=1}^{\infty} \bigcup \limits_{k \geq n}^{\infty} A_k)^C = \bigcup \limits_{n=1}^{\infty} \bigcap \limits_{k \geq n}^{\infty} A_k^C = \lim \inf A_k^C$\\
		$P(A^C) \leq \sum \limits_{n=1}^{\infty}P(\bigcap \limits_{k \geq n} A_k^C)$ (субаддитивность)\\
		Фиксируем $n \in \N$. $P(\bigcap \limits_{k=n}^N A_k^C) \xrightarrow{N \to \infty} P(\bigcap \limits_{k=n}^{\infty} A_k^C)$ по непрерывности вероятностной меры.\\
		$P(\bigcap \limits_{k=n}^N A_k^C) = \prod \limits_{k=n}^N P(A_k^C)$ - из леммы $\{A_n, \dots, A_N\}$ - независимы $\implies \{A_n^C, \dots, A_N^C\}$ - независимы.\\
		$P(\bigcap \limits_{k=n}^N A_k^C) = \prod \limits_{k=n}^N P(A_k^C) = \prod \limits_{k=n}^N (1-P(A_k)) \overset{\makebox[30pt]{\mbox{\normalfont\tiny $1-x \leq e^x$}}}{\leq}  \prod \limits_{k=n}^N e^{-P(A_k)}=$\\
		$ = e^{-\sum \limits_{k=n}^N P(A_k)}  \xrightarrow{N \to \infty} 0$ (так как ряд расходится).
	\end{enumerate}
\end{proof}